% v2-acmsmall-sample.tex, dated March 6 2012
% This is a sample file for ACM small trim journals
%
% Compilation using 'acmsmall.cls' - version 1.3 (March 2012), Aptara Inc.
% (c) 2010 Association for Computing Machinery (ACM)
%
% Questions/Suggestions/Feedback should be addressed to => "acmtexsupport@aptaracorp.com".
% Users can also go through the FAQs available on the journal's submission webpage.
%
% Steps to compile: latex, bibtex, latex latex
%
% For tracking purposes => this is v1.3 - March 2012

\documentclass[prodmode,acmtodaes]{acmsmall} % Aptara syntax

% Package to generate and customize Algorithm as per ACM style
\usepackage[ruled]{algorithm2e}
\renewcommand{\algorithmcfname}{ALGORITHM}
\SetAlFnt{\small}
\SetAlCapFnt{\small}
\SetAlCapNameFnt{\small}
\SetAlCapHSkip{0pt}
\IncMargin{-\parindent}

% Metadata Information
\acmVolume{???}
\acmNumber{??}
\acmArticle{?}
\acmYear{2016}
\acmMonth{9}

% Copyright
%\setcopyright{acmcopyright}
%\setcopyright{acmlicensed}
%\setcopyright{rightsretained}
%\setcopyright{usgov}
%\setcopyright{usgovmixed}
%\setcopyright{cagov}
%\setcopyright{cagovmixed}

% DOI
\doi{0000001.0000001}

%ISSN
\issn{1234-56789}

% Document starts
\begin{document}

% Page heads
\markboth{G. Zhou et al.}{A Multifrequency MAC Specially Designed for WSN Applications}

% Title portion
\title{Taint Analysis of Microarchitecture Synthesis Software}
\author{ZHIYANG ONG
\affil{Texas A\&M University}
%	YAFENG WU
%	\affil{University of Virginia}
%	TING YAN
%	\affil{Eaton Innovation Center}
%	TIAN HE
%	\affil{University of Minnesota}
%	CHENGDU HUANG
%	\affil{Google}
%	JOHN A. STANKOVIC
%	\affil{University of Virginia}
%	TAREK F. ABDELZAHER
%	\affil{University of Illinois at Urbana-Champaign}
}
% NOTE! Affiliations placed here should be for the institution where the
%       BULK of the research was done. If the author has gone to a new
%       institution, before publication, the (above) affiliation should NOT be changed.
%       The authors 'current' address may be given in the "Author's addresses:" block (below).
%       So for example, Mr. Abdelzaher, the bulk of the research was done at UIUC, and he is
%       currently affiliated with NASA.

%	\begin{abstract}
%	Multifrequency media access control has been well understood in
%	general wireless ad hoc networks, while in wireless sensor networks,
%	researchers still focus on single frequency solutions. In wireless
%	sensor networks, each device is typically equipped with a single
%	radio transceiver and applications adopt much smaller packet sizes
%	compared to those in general wireless ad hoc networks. Hence, the
%	multifrequency MAC protocols proposed for general wireless ad hoc
%	networks are not suitable for wireless sensor network applications,
%	which we further demonstrate through our simulation experiments. In
%	this article, we propose MMSN, which takes advantage of
%	multifrequency availability while, at the same time, takes into
%	consideration the restrictions of wireless sensor networks. Through
%	extensive experiments, MMSN exhibits the prominent ability to utilize
%	parallel transmissions among neighboring nodes. When multiple physical
%	frequencies are available, it also achieves increased energy
%	efficiency, demonstrating the ability to work against radio
%	interference and the tolerance to a wide range of measured time
%	synchronization errors.
%	\end{abstract}


%
% The code below should be generated by the tool at
% http://dl.acm.org/ccs.cfm
% Please copy and paste the code instead of the example below. 
%
%	\begin{CCSXML}
%	<ccs2012>
%	 <concept>
%	  <concept_id>10010520.10010553.10010562</concept_id>
%	  <concept_desc>Computer systems organization~Embedded systems</concept_desc>
%	  <concept_significance>500</concept_significance>
%	 </concept>
%	 <concept>
%	  <concept_id>10010520.10010575.10010755</concept_id>
%	  <concept_desc>Computer systems organization~Redundancy</concept_desc>
%	  <concept_significance>300</concept_significance>
%	 </concept>
%	 <concept>
%	  <concept_id>10010520.10010553.10010554</concept_id>
%	  <concept_desc>Computer systems organization~Robotics</concept_desc>
%	  <concept_significance>100</concept_significance>
%	 </concept>
%	 <concept>
%	  <concept_id>10003033.10003083.10003095</concept_id>
%	  <concept_desc>Networks~Network reliability</concept_desc>
%	  <concept_significance>100</concept_significance>
%	 </concept>
%	</ccs2012>  
%	\end{CCSXML}
%	
%	\ccsdesc[500]{Computer systems organization~Embedded systems}
%	\ccsdesc[300]{Computer systems organization~Redundancy}
%	\ccsdesc{Computer systems organization~Robotics}
%	\ccsdesc[100]{Networks~Network reliability}

%
% End generated code
%

% We no longer use \terms command
%\terms{Design, Algorithms, Performance}

%	\keywords{Wireless sensor networks, media access control,
%	multi-channel, radio interference, time synchronization}

%	\acmformat{Zhiyang Ong, 2016. Taint Analysis of Microarchitecture Synthesis Software.}
% At a minimum you need to supply the author names, year and a title.
% IMPORTANT:
% Full first names whenever they are known, surname last, followed by a period.
% In the case of two authors, 'and' is placed between them.
% In the case of three or more authors, the serial comma is used, that is, all author names
% except the last one but including the penultimate author's name are followed by a comma,
% and then 'and' is placed before the final author's name.
% If only first and middle initials are known, then each initial
% is followed by a period and they are separated by a space.
% The remaining information (journal title, volume, article number, date, etc.) is 'auto-generated'.

%	\begin{bottomstuff}
%	%	This work is supported by the National Science Foundation, under
%	%	grant CNS-0435060, grant CCR-0325197 and grant EN-CS-0329609.
%	
%	Author's addresses: Z. Ong, Department of Electrical and Computer Engineering,
%	Dwight Look College of Engineering,
%	Texas A\&M University
%	\end{bottomstuff}

\maketitle

Microarchitecture synthesis software is used to generate cycle-accurate/RTL designs of microarchitectures. 

Taint analysis has been used to determine security loopholes of software. 

Project Goal: Apply taint analysis to determine the security loopholes of the microarchitecture synthesis software. 

Questions
Can I force a finite state machine (FSM) to enter a non-existent/invalid state?
Can I force the processor to write back to memory out of order?
Can I degrade the performance of many-core processors?




Benchmarks (now): %\vspace{-0.2cm}
		\begin{enumerate} %\itemsep -2pt
		\item Rocket chip: https://github.com/ucb-bar/rocket-chip
		\item OpenSoCFabric: https://github.com/ucb-bar/OpenSoCFabric
		\item chisel-torture: https://github.com/ucb-bar/chisel-torture








Insert text here.


% Start of "Sample References" section

%	\section{Typical references in new ACM Reference Format}
%	A paginated journal article \cite{Abril07}, an enumerated
%	journal article \cite{Cohen07}, a reference to an entire issue \cite{JCohen96},
%	a monograph (whole book) \cite{Kosiur01}, a monograph/whole book in a series (see 2a in spec. document)
%	\cite{Harel79}, a divisible-book such as an anthology or compilation \cite{Editor00}
%	followed by the same example, however we only output the series if the volume number is given
%	\cite{Editor00a} (so Editor00a's series should NOT be present since it has no vol. no.),
%	a chapter in a divisible book \cite{Spector90}, a chapter in a divisible book
%	in a series \cite{Douglass98}, a multi-volume work as book \cite{Knuth97},
%	an article in a proceedings (of a conference, symposium, workshop for example)
%	(paginated proceedings article) \cite{Andler79}, a proceedings article
%	with all possible elements \cite{Smith10}, an example of an enumerated
%	proceedings article \cite{VanGundy07},
%	an informally published work \cite{Harel78}, a doctoral dissertation \cite{Clarkson85},
%	a master's thesis: \cite{anisi03}, an online document / world wide web resource \cite{Thornburg01}, \cite{Ablamowicz07},
%	\cite{Poker06}, a video game (Case 1) \cite{Obama08} and (Case 2) \cite{Novak03}
%	and \cite{Lee05} and (Case 3) a patent \cite{JoeScientist001},
%	work accepted for publication \cite{rous08}, 'YYYYb'-test for prolific author
%	\cite{SaeediMEJ10} and \cite{SaeediJETC10}. Other cites might contain
%	'duplicate' DOI and URLs (some SIAM articles) \cite{Kirschmer:2010:AEI:1958016.1958018}.
%	Boris / Barbara Beeton: multi-volume works as books
%	\cite{MR781536} and \cite{MR781537}.

% Appendix
%	\appendix
%	\section*{APPENDIX}
%	\setcounter{section}{1}
%	In this appendix, we measure
%	the channel switching time of Micaz [CROSSBOW] sensor devices.
%	In our experiments, one mote alternatingly switches between Channels
%	11 and 12. Every time after the node switches to a channel, it sends
%	out a packet immediately and then changes to a new channel as soon
%	as the transmission is finished. We measure the
%	number of packets the test mote can send in 10 seconds, denoted as
%	$N_{1}$. In contrast, we also measure the same value of the test
%	mote without switching channels, denoted as $N_{2}$. We calculate
%	the channel-switching time $s$ as
%	\begin{eqnarray}%
%	s=\frac{10}{N_{1}}-\frac{10}{N_{2}}. \nonumber
%	\end{eqnarray}%
%	By repeating the experiments 100 times, we get the average
%	channel-switching time of Micaz motes: 24.3$\mu$s.
%	
%	\appendixhead{ZHOU}

% Acknowledgments
%	\begin{acks}
%	The authors would like to thank Dr. Maura Turolla of Telecom
%	Italia for providing specifications about the application scenario.
%	\end{acks}

% Bibliography
\bibliographystyle{ACM-Reference-Format-Journals}
\bibliography{acmsmall-sample-bibfile}
                             % Sample .bib file with references that match those in
                             % the 'Specifications Document (V1.5)' as well containing
                             % 'legacy' bibs and bibs with 'alternate codings'.
                             % Gerry Murray - March 2012

% History dates
%\received{February 2007}{March 2009}{June 2009}

% Electronic Appendix
%	\elecappendix
%	
%	\medskip
%	
%	\section{This is an example of Appendix section head}
%	
%	Channel-switching time is measured as the time length it takes for
%	motes to successfully switch from one channel to another. This
%	parameter impacts the maximum network throughput, because motes
%	cannot receive or send any packet during this period of time, and it
%	also affects the efficiency of toggle snooping in MMSN, where motes
%	need to sense through channels rapidly.
%	
%	By repeating experiments 100 times, we get the average
%	channel-switching time of Micaz motes: 24.3 $\mu$s. We then conduct
%	the same experiments with different Micaz motes, as well as
%	experiments with the transmitter switching from Channel 11 to other
%	channels. In both scenarios, the channel-switching time does not have
%	obvious changes. (In our experiments, all values are in the range of
%	23.6 $\mu$s to 24.9 $\mu$s.)
%	
%	\section{Appendix section head}
%	
%	The primary consumer of energy in WSNs is idle listening. The key to
%	reduce idle listening is executing low duty-cycle on nodes. Two
%	primary approaches are considered in controlling duty-cycles in the
%	MAC layer.

\end{document}
% End of v2-acmsmall-sample.tex (March 2012) - Gerry Murray, ACM


